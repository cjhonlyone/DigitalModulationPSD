% Options for packages loaded elsewhere
\PassOptionsToPackage{unicode}{hyperref}
\PassOptionsToPackage{hyphens}{url}
%
\documentclass[
]{article}
\usepackage{lmodern}
\usepackage{amssymb,amsmath}
\usepackage{ifxetex,ifluatex}
\ifnum 0\ifxetex 1\fi\ifluatex 1\fi=0 % if pdftex
  \usepackage[T1]{fontenc}
  \usepackage[utf8]{inputenc}
  \usepackage{textcomp} % provide euro and other symbols
\else % if luatex or xetex
  \usepackage{unicode-math}
  \defaultfontfeatures{Scale=MatchLowercase}
  \defaultfontfeatures[\rmfamily]{Ligatures=TeX,Scale=1}
\fi
% Use upquote if available, for straight quotes in verbatim environments
\IfFileExists{upquote.sty}{\usepackage{upquote}}{}
\IfFileExists{microtype.sty}{% use microtype if available
  \usepackage[]{microtype}
  \UseMicrotypeSet[protrusion]{basicmath} % disable protrusion for tt fonts
}{}
\makeatletter
\@ifundefined{KOMAClassName}{% if non-KOMA class
  \IfFileExists{parskip.sty}{%
    \usepackage{parskip}
  }{% else
    \setlength{\parindent}{0pt}
    \setlength{\parskip}{6pt plus 2pt minus 1pt}}
}{% if KOMA class
  \KOMAoptions{parskip=half}}
\makeatother
\usepackage{xcolor}
\IfFileExists{xurl.sty}{\usepackage{xurl}}{} % add URL line breaks if available
\IfFileExists{bookmark.sty}{\usepackage{bookmark}}{\usepackage{hyperref}}
\hypersetup{
  hidelinks,
  pdfcreator={LaTeX via pandoc}}
\urlstyle{same} % disable monospaced font for URLs
\usepackage{color}
\usepackage{fancyvrb}
\newcommand{\VerbBar}{|}
\newcommand{\VERB}{\Verb[commandchars=\\\{\}]}
\DefineVerbatimEnvironment{Highlighting}{Verbatim}{commandchars=\\\{\}}
% Add ',fontsize=\small' for more characters per line
\newenvironment{Shaded}{}{}
\newcommand{\AlertTok}[1]{\textcolor[rgb]{1.00,0.00,0.00}{\textbf{#1}}}
\newcommand{\AnnotationTok}[1]{\textcolor[rgb]{0.38,0.63,0.69}{\textbf{\textit{#1}}}}
\newcommand{\AttributeTok}[1]{\textcolor[rgb]{0.49,0.56,0.16}{#1}}
\newcommand{\BaseNTok}[1]{\textcolor[rgb]{0.25,0.63,0.44}{#1}}
\newcommand{\BuiltInTok}[1]{#1}
\newcommand{\CharTok}[1]{\textcolor[rgb]{0.25,0.44,0.63}{#1}}
\newcommand{\CommentTok}[1]{\textcolor[rgb]{0.38,0.63,0.69}{\textit{#1}}}
\newcommand{\CommentVarTok}[1]{\textcolor[rgb]{0.38,0.63,0.69}{\textbf{\textit{#1}}}}
\newcommand{\ConstantTok}[1]{\textcolor[rgb]{0.53,0.00,0.00}{#1}}
\newcommand{\ControlFlowTok}[1]{\textcolor[rgb]{0.00,0.44,0.13}{\textbf{#1}}}
\newcommand{\DataTypeTok}[1]{\textcolor[rgb]{0.56,0.13,0.00}{#1}}
\newcommand{\DecValTok}[1]{\textcolor[rgb]{0.25,0.63,0.44}{#1}}
\newcommand{\DocumentationTok}[1]{\textcolor[rgb]{0.73,0.13,0.13}{\textit{#1}}}
\newcommand{\ErrorTok}[1]{\textcolor[rgb]{1.00,0.00,0.00}{\textbf{#1}}}
\newcommand{\ExtensionTok}[1]{#1}
\newcommand{\FloatTok}[1]{\textcolor[rgb]{0.25,0.63,0.44}{#1}}
\newcommand{\FunctionTok}[1]{\textcolor[rgb]{0.02,0.16,0.49}{#1}}
\newcommand{\ImportTok}[1]{#1}
\newcommand{\InformationTok}[1]{\textcolor[rgb]{0.38,0.63,0.69}{\textbf{\textit{#1}}}}
\newcommand{\KeywordTok}[1]{\textcolor[rgb]{0.00,0.44,0.13}{\textbf{#1}}}
\newcommand{\NormalTok}[1]{#1}
\newcommand{\OperatorTok}[1]{\textcolor[rgb]{0.40,0.40,0.40}{#1}}
\newcommand{\OtherTok}[1]{\textcolor[rgb]{0.00,0.44,0.13}{#1}}
\newcommand{\PreprocessorTok}[1]{\textcolor[rgb]{0.74,0.48,0.00}{#1}}
\newcommand{\RegionMarkerTok}[1]{#1}
\newcommand{\SpecialCharTok}[1]{\textcolor[rgb]{0.25,0.44,0.63}{#1}}
\newcommand{\SpecialStringTok}[1]{\textcolor[rgb]{0.73,0.40,0.53}{#1}}
\newcommand{\StringTok}[1]{\textcolor[rgb]{0.25,0.44,0.63}{#1}}
\newcommand{\VariableTok}[1]{\textcolor[rgb]{0.10,0.09,0.49}{#1}}
\newcommand{\VerbatimStringTok}[1]{\textcolor[rgb]{0.25,0.44,0.63}{#1}}
\newcommand{\WarningTok}[1]{\textcolor[rgb]{0.38,0.63,0.69}{\textbf{\textit{#1}}}}
\usepackage{longtable,booktabs}
% Correct order of tables after \paragraph or \subparagraph
\usepackage{etoolbox}
\makeatletter
\patchcmd\longtable{\par}{\if@noskipsec\mbox{}\fi\par}{}{}
\makeatother
% Allow footnotes in longtable head/foot
\IfFileExists{footnotehyper.sty}{\usepackage{footnotehyper}}{\usepackage{footnote}}
\makesavenoteenv{longtable}
\setlength{\emergencystretch}{3em} % prevent overfull lines
\providecommand{\tightlist}{%
  \setlength{\itemsep}{0pt}\setlength{\parskip}{0pt}}
\setcounter{secnumdepth}{-\maxdimen} % remove section numbering


\usepackage[UTF8]{ctex}
\usepackage{listings}
\usepackage{fontspec} % 定制字体
% \newfontfamily\menlo{Menlo}
\usepackage{xcolor} % 定制颜色
\definecolor{mygreen}{rgb}{0,0.6,0}
\definecolor{mygray}{rgb}{0.5,0.5,0.5}
\definecolor{mymauve}{rgb}{0.58,0,0.82}
\lstdefinestyle{c-style}
{
    language=c,
    basicstyle=\footnotesize\ttfamily,  % size of fonts used for the code
    keywordstyle=\color{blue},     % keyword style
    % identifierstyle=\color{red},
    commentstyle=\color{mygreen},  % comment style
    frame=single,
    backgroundcolor=\color{white},      % choose the background color
    columns=fullflexible,
    tabsize=4,
    breaklines=true,               % automatic line breaking only at whitespace
    captionpos=b,                  % sets the caption-position to bottom
    escapeinside={\%*}{*)},        % if you want to add LaTeX within your code
    stringstyle=\color{mymauve}\ttfamily,  % string literal style
    rulesepcolor=\color{red!20!green!20!blue!20}
}

\definecolor{vgreen}{RGB}{104,180,104}
\definecolor{vblue}{RGB}{49,49,255}
\definecolor{vorange}{RGB}{255,143,102}
\definecolor{mygreen}{RGB}{28,172,0} % color values Red, Green, Blue
\definecolor{mylilas}{RGB}{170,55,241}

\lstdefinestyle{verilog-style}
{
    language=Verilog,
    basicstyle=\footnotesize\ttfamily,
    keywordstyle=\color{vblue},
    identifierstyle=\color{black},
    commentstyle=\color{vgreen},
    frame=single,
    % numbers=left,
    % numberstyle=\tiny\color{black},
    % numbersep=10pt,
    tabsize=4,
    moredelim=*[s][\colorIndex]{[}{]},
    literate=*{:}{:}1
}

\lstdefinestyle{Matlab-style}
{
	 language=Matlab,%
    %basicstyle=\color{red},
    breaklines=true,%
    morekeywords={matlab2tikz},
    keywordstyle=\color{blue},%
    morekeywords=[2]{1}, keywordstyle=[2]{\color{black}},
    identifierstyle=\color{black},%
    stringstyle=\color{mylilas},
    commentstyle=\color{mygreen},%
    showstringspaces=false,%without this there will be a symbol in the places where there is a space
    numbers=left,%
    numberstyle={\tiny \color{black}},% size of the numbers
    numbersep=9pt, % this defines how far the numbers are from the text
    emph=[1]{for,end,break},emphstyle=[1]\color{red} %some words to emphasise
    %emph=[2]{word1,word2}, emphstyle=[2]{style},    
}

\makeatletter
\newcommand*\@lbracket{[}
\newcommand*\@rbracket{]}
\newcommand*\@colon{:}
\newcommand*\colorIndex{%
    \edef\@temp{\the\lst@token}%
    \ifx\@temp\@lbracket \color{black}%
    \else\ifx\@temp\@rbracket \color{black}%
    \else\ifx\@temp\@colon \color{black}%
    \else \color{vorange}%
    \fi\fi\fi
}
\makeatother

\usepackage{trace}
\usepackage{url}
\usepackage{hyperref}
\hypersetup{
    colorlinks=true,
    linkcolor=black,
    filecolor=blue,      
    urlcolor=blue,
    citecolor=cyan,
}
\title{\huge \textbf{调制信号功率谱分析}}
\author{曹嘉辉}
\date{Edit 2022.6.30}
\begin{document}

\maketitle

github page: \url{https://github.com/cjhonlyone}
% \hypertarget{header-n0}{%
% \section{CRC校验一探究竟}\label{header-n0}}

\tableofcontents

\newpage
\hypertarget{header-n3}{%
\subsection{1. 原理}\label{header-n3}}

\hypertarget{header-n4}{%
\subsubsection{1.1 随机过程}\label{header-n4}}

\hypertarget{header-n5}{%
\subsubsection{1.2 Gambler's Ruin Problem(赌徒破产问题)}\label{header-n5}}
庄家输掉所有筹码的概率\\
庄家有$n$个筹码,每次有概率$p$赢得一个筹码,或者概率$q(q=1-p)$输掉一个筹码。庄家输掉所有钱后,即终止游戏。假设各次赌博都是独立的,求庄家把所有筹码输光的概率。\\
设$f[n], n=0,1,2...\infty$为庄家把拥有的n个筹码全部输掉的的概率。\\
那么有以下两个临界值:\\
$f[0]=1$(初始筹码为0时,代表输光了,概率为1)\\
$f[\infty]=0$(初始筹码为无限时,不可能输光,概率为0)\\
当前筹码的个数只跟前一次筹码的个数相关,则有递推公式,\\
\begin{equation}
\begin{aligned}
f[n] = pf[n+1] + qf[n-1]\\
\end{aligned}
\end{equation}
已知$p+q=1$,利用高中的数列求解方法有,
\begin{equation}
\begin{aligned}
pf[n] + qf[n] &= pf[n+1] + qf[n-1]\\
pf[n+1] - pf[n] &= qf[n] - qf[n-1]\\
f[n+1] - f[n] &= \frac{q}{p}(f[n] - f[n-1])\\
\end{aligned}
\end{equation}
可以写出\\
\begin{equation}
\begin{aligned}
f[n] - f[n-1] &= \frac{q}{p}(f[n-1] - f[n-2])\\
f[n-1] - f[n-2] &= \frac{q}{p}(f[n-2] - f[n-3])\\
\vdots\\
f[3] - f[2] &= \frac{q}{p}(f[2] - f[1])\\
f[2] - f[1] &= \frac{q}{p}(f[1] - f[0])\\
\end{aligned}
\end{equation}
根据递推关系有\\
\begin{equation}
\begin{aligned}
f[n] - f[n-1] &= (\frac{q}{p})^{n-1}(f[1] - f[0])\\
f[n-1] - f[n-2] &= (\frac{q}{p})^{n-2}(f[1] - f[0])\\
&\vdots\\
f[3] - f[2] &= (\frac{q}{p})^{2}(f[1] - f[0])\\
f[2] - f[1] &= \frac{q}{p}(f[1] - f[0])\\
\end{aligned}
\end{equation}
将上面一组式子左右两边相加\\
\begin{equation}
\begin{aligned}
f[n] = f[0]+(1 + \frac{q}{p} +  (\frac{q}{p})^{2} + \cdots +(\frac{q}{p})^{n-1})(f[1] - f[0])\\
\end{aligned}
\end{equation}
利用等比数列求和公式与$f[0]=1$\\
\begin{equation}
f[n] =\left\{
\begin{aligned}
&1+\frac{1-(\frac{q}{p})^n}{1-\frac{q}{p}}(f[1] - 1), &\frac{q}{p} \neq 1 \\
&1+n(f[1]-1), &\frac{q}{p} = 1 \\
\end{aligned}
\right.
\end{equation}
因为$f[\infty]=0$,可解得\\
\begin{equation}
f[1] =\left\{
\begin{aligned}
&\frac{q}{p}, &\frac{q}{p} < 1 \\
&1, &\frac{q}{p} \geq 1 \\
\end{aligned}
\right.
\end{equation}
带入上式有\\
\begin{equation}
f[n] =\left\{
\begin{aligned}
&(\frac{q}{p})^n, & q<p\\
&1, & q \geq p \\
\end{aligned}
\right.
\end{equation}
\begin{itemize}
\item
  当 $q < p$ 时,$p$相同时,庄家钱越多,输光筹码的概率越小
\item
  当 $q \geq p$ 时,庄家必然输光筹码
\end{itemize}

赌徒赢得N个筹码的概率\\
一个赌徒初始时有$n$个筹码,每次有概率$p$赢得一个筹码,或者概率$q(q=1-p)$输掉一个筹码。赌徒赢得$N$个筹码后,或者输掉所有钱后,即终止游戏。假设各次赌博都是独立的,求赌徒在输掉初始筹码前赢得$N$个筹码的概率。\\
设$f[n], n=0,1,2...\infty$为赌徒在输掉初始筹码$n$前,赢得$N$个筹码的概率。\\
那么有以下两个临界值:\\
$f[0]=0$(初始筹码为0时,无法参与赌博,概率为0)\\
$f[N]=1$(初始筹码为无限时,不可能输光,概率为1)\\
当前筹码的个数只跟前一次筹码的个数相关,则有递推公式,\\
\begin{equation}
\begin{aligned}
f[n] = pf[n+1] + qf[n-1]\\
\end{aligned}
\end{equation}
已知$p+q=1$,利用高中的数列求解方法有,
\begin{equation}
\begin{aligned}
pf[n] + qf[n] &= pf[n+1] + qf[n-1]\\
pf[n+1] - pf[n] &= qf[n] - qf[n-1]\\
f[n+1] - f[n] &= \frac{q}{p}(f[n] - f[n-1])\\
\end{aligned}
\end{equation}
可以写出\\
\begin{equation}
\begin{aligned}
f[n] - f[n-1] &= \frac{q}{p}(f[n-1] - f[n-2])\\
f[n-1] - f[n-2] &= \frac{q}{p}(f[n-2] - f[n-3])\\
\vdots\\
f[3] - f[2] &= \frac{q}{p}(f[2] - f[1])\\
f[2] - f[1] &= \frac{q}{p}(f[1] - f[0])\\
\end{aligned}
\end{equation}
根据递推关系有\\
\begin{equation}
\begin{aligned}
f[n] - f[n-1] &= (\frac{q}{p})^{n-1}(f[1] - f[0])\\
f[n-1] - f[n-2] &= (\frac{q}{p})^{n-2}(f[1] - f[0])\\
&\vdots\\
f[3] - f[2] &= (\frac{q}{p})^{2}(f[1] - f[0])\\
f[2] - f[1] &= \frac{q}{p}(f[1] - f[0])\\
\end{aligned}
\end{equation}
将上面一组式子左右两边相加\\
\begin{equation}
\begin{aligned}
f[n] = f[0]+(1 + \frac{q}{p} +  (\frac{q}{p})^{2} + \cdots +(\frac{q}{p})^{n-1})(f[1] - f[0])\\
\end{aligned}
\end{equation}
利用等比数列求和公式与$f[0]=0$\\
\begin{equation}
f[n] =\left\{
\begin{aligned}
&\frac{1-(\frac{q}{p})^n}{1-\frac{q}{p}} f[1], &\frac{q}{p} \neq 1 \\
&nf[1], &\frac{q}{p} = 1 \\
\end{aligned}
\right.
\end{equation}
因为$f[N]=1$,可解得\\
\begin{equation}
f[1] =\left\{
\begin{aligned}
&\frac{1-\frac{q}{p}}{1-(\frac{q}{p})^N}, &\frac{q}{p} \neq 1 \\
&\frac{1}{N}, &\frac{q}{p} = 1 \\
\end{aligned}
\right.
\end{equation}
带入上式有\\
\begin{equation}
f[n] =\left\{
\begin{aligned}
&\frac{1-(\frac{q}{p})^n}{1-(\frac{q}{p})^N}, &\frac{q}{p} \neq 1 \\
&\frac{n}{N}, &\frac{q}{p} = 1 \\
\end{aligned}
\right.
\end{equation}
当$N\to\infty$时\\
\begin{equation}
f[n] =\left\{
\begin{aligned}
&1-(\frac{q}{p})^n, & q<p\\
&0, & q \geq p \\
\end{aligned}
\right.
\end{equation}
\begin{itemize}
\item
  当 $q < p$ 时,$p$相同时,本金越多,赌徒赢的概率越大
\item
  当 $q \geq p$ 时,赌徒无法完成目标
\end{itemize}

\hypertarget{header-n12}{%
\subsubsection{1.2 循环自相关函数}\label{header-n12}}

\hypertarget{header-n13}{%
\subsubsection{1.3 功率谱密度与自相关函数}\label{header-n13}}

\hypertarget{header-n16}{%
\subsection{2 一些典型信号的功率谱密度}\label{header-n16}}

\hypertarget{header-n21}{%
\subsubsection{2.1 离散时间脉冲串}\label{header-n21}}
$a_k$为离散时间序列,$-\infty<k<\infty$。\\
离散冲击函数(Kronecker Delta Function)为
\begin{equation}
\delta[n] =\left\{
\begin{aligned}
&1, n=0, \\
&0, n=others.
\end{aligned}
\right.
\end{equation}

离散时间脉冲串的时域表示如下
\begin{equation}
\begin{aligned}
x[n] &= \sum\limits_{k=-\infty}^{\infty} a_k \delta[n+k] \\
\end{aligned}
\end{equation}
根据自相关函数的定义有
\begin{equation}
\begin{aligned}
R[k] &= E[x[n]x[n-k]]\\
&= E[\sum\limits_{m=-\infty}^{\infty} a_m \delta[n-m] \sum\limits_{j=-\infty}^{\infty} a_j \delta[n-j+k]]\\
&= E[\sum\limits_{m=-\infty}^{\infty} \sum\limits_{j=-\infty}^{\infty} a_m a_j \delta[n-m] \delta[n-j+k]]\\
&\overset{j=m+k}= E[\sum\limits_{m=-\infty}^{\infty} a_m a_{m+k} \delta[n-m]]\\
&= E[a_n a_{n-k}]\\
\end{aligned}
\end{equation}
时间平均代替集平均 \\
\begin{equation}
\begin{aligned}
R[k] &= \lim\limits_{N \to \infty} \frac{1}{2N}\sum\limits_{n=-N}^{N} a_n a_{n-k}\\
\end{aligned}
\end{equation}

\hypertarget{header-n22}{%
\subsubsection{2.2 连续时间脉冲串}\label{header-n22}}

$a_k$为离散时间序列,$-\infty<k<\infty$。\\
连续冲击函数(Dirac Delta Function)为
\begin{equation}
\delta(t) =\left\{
\begin{aligned}
&\infty, t=0, \\
&0, t=others.
\end{aligned}
\right.
\end{equation}
连续时间脉冲串的时域表示如下,$t_0$是在$[0,T)$内均匀分布的随机变量。\\
\begin{equation}
\begin{aligned}
x(t) &= \sum\limits_{k=-\infty}^{\infty} a_k \delta(t-kT_b-t_0) \\
\end{aligned}
\end{equation}
根据自相关函数的定义有\\
\begin{equation}
\begin{aligned}
R(\tau) &= E[x(t)x(t+\tau)]\\
&= \lim\limits_{N \to \infty} \frac{1}{2NT_b} \int_{-NT_b}^{NT_b} x(t)x(t+\tau) dt\\
&= \lim\limits_{N \to \infty} \frac{1}{2NT_b} \int_{-NT_b}^{NT_b} \sum\limits_{n=-N}^{N} a_n \delta(t-nT_b) \sum\limits_{j=-N}^{N} a_j \delta(t-jT_b+\tau)dt\\
&= \lim\limits_{N \to \infty} \frac{1}{2NT_b} \int_{-NT_b}^{NT_b} \sum\limits_{n=-N}^{N} \sum\limits_{j=-N}^{N} a_n a_j \delta(t-nT_b) \delta(t-jT_b+\tau)dt\\
&= \lim\limits_{N \to \infty} \frac{1}{2NT_b} \sum\limits_{n=-N}^{N} \sum\limits_{j=-N}^{N} a_n a_j \int_{-NT_b}^{NT_b} \delta(t-nT_b) \delta(t-jT_b+\tau) dt\\
&= \lim\limits_{N \to \infty} \frac{1}{2NT_b} \sum\limits_{n=-N}^{N} \sum\limits_{j=-N}^{N} a_n a_j \delta(\tau-(j-n)T_b)\\
&\overset{j=n+k}= \lim\limits_{N \to \infty} \frac{1}{2NT_b} \sum\limits_{n=-N}^{N} \sum\limits_{k=-N}^{N} a_n a_{n+k} \delta(\tau-kT_b)\\
&= \frac{1}{T_b} \sum\limits_{k=-\infty}^{\infty} \lim\limits_{N \to \infty} \frac{1}{2N} \sum\limits_{n=-N}^{N} a_n a_{n+k} \delta(\tau-kT_b)\\
&= \frac{1}{T_b} \sum\limits_{k=-\infty}^{\infty} \rho[k] \delta(\tau-kT_b)\\
\end{aligned}
\end{equation}
在这里
\begin{equation}
\begin{aligned}
\rho[k] &= \lim\limits_{N \to \infty} \frac{1}{2N} \sum\limits_{n=-N}^{N} a_n a_{n+k}\\
&= E[a_n a_{n-k}]
\end{aligned}
\end{equation}
再换一种推导方式
\begin{equation}
\begin{aligned}
R(\tau) &= E[x(t)x(t+\tau)]\\
&= E[\sum\limits_{n=-\infty}^{\infty} a_n \delta(t-nT_b) \sum\limits_{j=-\infty}^{\infty} a_j \delta(t-jT_b+\tau) ] \\
&= E[\sum\limits_{n=-\infty}^{\infty} \sum\limits_{j=-\infty}^{\infty} a_n a_j \delta(t-nT_b) \delta(t-jT_b+\tau) ] \\
&= E[\sum\limits_{k=-\infty}^{\infty} \sum\limits_{n=-\infty}^{\infty} \sum\limits_{j=-\infty}^{\infty} a_n a_j \delta(t-nT_b) \delta(t-jT_b+kT_b) \delta(\tau-kT_b)] \\
&\overset{j=n+k}= E[\sum\limits_{k=-\infty}^{\infty} \sum\limits_{n=-\infty}^{\infty} a_n a_{n+k} \delta(t-nT_b) \delta(\tau-kT_b)] \\
&= \lim\limits_{N \to \infty} \frac{1}{2NT_b} \int_{-NT_b}^{NT_b} \sum\limits_{k=-N}^{N} \sum\limits_{n=-N}^{N} a_n a_{n+k} \delta(t-nT_b) \delta(\tau-kT_b) dt \\
&= \frac{1}{T_b} \sum\limits_{k=-\infty}^{\infty} \lim\limits_{N \to \infty} \frac{1}{2N} \sum\limits_{n=-N}^{N} a_n a_{n+k} \delta(\tau-kT_b)\\
&= \frac{1}{T_b} \sum\limits_{k=-\infty}^{\infty} \rho[k] \delta(\tau-kT_b)\\
\end{aligned}
\end{equation}

\begin{equation}
\begin{aligned}
\rho[0] &= E[a_n^2]\\
&= D[a_n] + E[a_n]^2\\
&= \sigma_A^2 + \mu_A^2\\
\end{aligned}
\end{equation}

还可以推导出$R(\tau)$的另一种表示形式,重新考虑自相关函数内部$n,j$的二重求和
\begin{equation}
\begin{aligned}
&E[\sum\limits_{n=-\infty}^{\infty} \sum\limits_{j=-\infty}^{\infty} a_n a_j \delta(t-nT_b-t_0) \delta(t-jT_b+\tau-t_0) ]\\
=&E[\sum\limits_{n=-\infty}^{\infty} a_n^2 \delta(t-nT_b-t_0) \delta(t-nT_b+\tau-t_0)]+\\
&E[\sum\limits_{n=-\infty}^{\infty} \sum\limits_{\substack{j=-\infty\\j\neq n}}^{\infty} a_n a_j \delta(t-nT_b-t_0) \delta(t-jT_b+\tau-t_0)]\\
=&\sum\limits_{n=-\infty}^{\infty} E[a_n^2] E[\delta(t-nT_b-t_0) \delta(t-nT_b+\tau-t_0)]+\\
&\sum\limits_{n=-\infty}^{\infty} \sum\limits_{\substack{j=-\infty\\j\neq n}}^{\infty} E[a_n a_j] E[\delta(t-nT_b-t_0) \delta(t-jT_b+\tau-t_0)]\\
\end{aligned}
\end{equation}
对于上式的第一项,求$t_0$在$[0,T_b)$之间的集平均有,
\begin{equation}
\begin{aligned}
&\sum\limits_{n=-\infty}^{\infty} E[a_n^2] E[\delta(t-nT_b-t_0) \delta(t-nT_b+\tau-t_0)]\\
=&E[a_n^2] \sum\limits_{n=-\infty}^{\infty} \frac{1}{T_b} \int_{0}^{T_b}\delta(t-nT_b-t_0) \delta(t-nT_b+\tau-t_0) d t_0\\
\end{aligned}
\end{equation}
令$u=t-nT_b+\tau-t_0$
\begin{equation}
\begin{aligned}
&= E[a_n^2] \frac{1}{T_b} \sum\limits_{n=-\infty}^{\infty} \int_{t-nT_b+\tau-T_b}^{t-nT_b+\tau}\delta(u-\tau) \delta(u) du\\
&= E[a_n^2] \frac{1}{T_b} \int_{-\infty}^{\infty}\delta(u-\tau) \delta(u) du\\
&= E[a_n^2] \frac{1}{T_b} \delta(\tau)\\
\end{aligned}
\end{equation}
对于第二项,求$t_0$在$[0,T_b)$之间的集平均有,
\begin{equation}
\begin{aligned}
&\sum\limits_{n=-\infty}^{\infty} \sum\limits_{\substack{j=-\infty\\j\neq n}}^{\infty} E[a_n a_j] E[\delta(t-nT_b-t_0) \delta(t-jT_b+\tau-t_0)]\\
=&E[a_n]^2 \sum\limits_{n=-\infty}^{\infty} \sum\limits_{\substack{j=-\infty\\j\neq n}}^{\infty} \frac{1}{T_b} \int_{0}^{T_b}\delta(t-nT_b-t_0) \delta(t-jT_b+\tau-t_0)) d t_0\\
\end{aligned}
\end{equation}
令$u=t-jT_b+\tau-t_0$
\begin{equation}
\begin{aligned}
&= E[a_n]^2 \frac{1}{T_b} \sum\limits_{n=-\infty}^{\infty} \sum\limits_{\substack{j=-\infty\\j\neq n}}^{\infty} \int_{t-jT_b+\tau-T_b}^{t-jT_b+\tau}\delta(u-\tau+(j-n)T_b) \delta(u) du\\
\end{aligned}
\end{equation}
因为积分内只与离散序号的差相关,令$m=j-n$
\begin{equation}
\begin{aligned}
&= E[a_n]^2 \frac{1}{T_b} \sum\limits_{n=-\infty}^{\infty} \sum\limits_{\substack{m=-\infty\\m\neq 0}}^{\infty} \int_{t-(m+n)T_b+\tau-T_b}^{t-(m+n)T_b+\tau} \delta(u-\tau+mT_b) \delta(u) du\\
&= E[a_n]^2 \frac{1}{T_b} \sum\limits_{\substack{m=-\infty\\m\neq 0}}^{\infty} \sum\limits_{n=-\infty}^{\infty} \int_{t-(m+n)T_b+\tau-T_b}^{t-(m+n)T_b+\tau} \delta(u-\tau+mT_b) \delta(u) du\\
&= E[a_n]^2 \frac{1}{T_b} \sum\limits_{\substack{m=-\infty\\m\neq 0}}^{\infty} \int_{-\infty}^{\infty}\delta(u-\tau+mT_b) \delta(u) du\\
&= E[a_n]^2 \frac{1}{T_b} \sum\limits_{\substack{m=-\infty\\m\neq 0}}^{\infty} \delta(\tau-mT_b)\\
\end{aligned}
\end{equation}
所以,$R(\tau)$可以表示为
\begin{equation}
\begin{aligned}
R(\tau) &= E[a_n^2]\frac{1}{T_b} \delta(\tau) + E[a_n]^2 \frac{1}{T_b} \sum\limits_{\substack{m=-\infty\\m\neq 0}}^{\infty} \delta(\tau-mT_b)\\
&= (E[a_n^2]- E[a_n]^2) \frac{1}{T_b} \delta(\tau) + E[a_n]^2 \frac{1}{T_b} \sum\limits_{\substack{m=-\infty}}^{\infty} \delta(\tau-mT_b)
\end{aligned}
\end{equation}
功率谱密度为\\
\begin{equation}
\begin{aligned}
S(f) &= \frac{1}{T_b} [(E[a_n^2]- E[a_n]^2) + E[a_n]^2 \frac{1}{T_b} \sum\limits_{\substack{k=-\infty}}^{\infty} \delta(f-\frac{k}{T_b})]\\
\end{aligned}
\end{equation}


\hypertarget{header-n23}{%
\subsubsection{2.3 脉冲串与码元的关系}\label{header-n23}}
连续时间脉冲串的时域表示如下
\begin{equation}
\begin{aligned}
x(t) &= \sum\limits_{k=-\infty}^{\infty} a_k \delta(t-kT_b) \\
\end{aligned}
\end{equation}
成型脉冲与冲击串卷积即可得到连续码元的波形\\
\begin{equation}
\begin{aligned}
y(t) &= p(t) \ast x(t) \\
&= \sum\limits_{k=-\infty}^{\infty} a_k p(t-kT_b) \\
\end{aligned}
\end{equation}
由随机过程通过线性系统前后的功率谱关系可知,
\begin{equation}
\begin{aligned}
S_y(f) &= 	{\left| S_p(f) \right|}^2 S_x(f) \\
\end{aligned}
\end{equation}

定义\\
\begin{equation}
\begin{aligned}
sinc(x) &= \frac{sin(x)}{x} \\
\end{aligned}
\end{equation}



对于非归零码(NRZ)一般有
\begin{equation}
p(t) =\left\{
\begin{aligned}
&1, & -T_b/2 < t < T_b/2\\
&0, & others \\
\end{aligned}
\right.
\end{equation}
\begin{equation}
\begin{aligned}
p(f) &= \frac{sin(\pi f T_b)}{\pi f}\\
&=  T_b sinc(\pi f T_b)\\
{\left| S_p(f) \right|}^2 & = T_b^2 sinc^2(\pi f T_b)
\end{aligned}
\end{equation}


对于归零码(RZ)一般有
\begin{equation}
p(t) =\left\{
\begin{aligned}
&1, & -T_b/4 < t < T_b/4\\
&0, & others \\
\end{aligned}
\right.
\end{equation}
\begin{equation}
\begin{aligned}
p(f) &= \frac{sin(\pi f T_b/2)}{\pi f}\\
&=  \frac{T_b}{2} sinc(\pi f T_b/2)\\
{\left| S_p(f) \right|}^2 & = \frac{T_b^2}{4} sinc^2(\pi f T_b/2)
\end{aligned}
\end{equation}

对于样本间独立的二进制序列${a_k}$,有
\begin{equation}
\rho[n] =\left\{
\begin{aligned}
&E[a_n^2], n = 0\\
&E^2[a_n], n \neq 0 \\
\end{aligned}
\right.
\end{equation}

\hypertarget{header-n24}{%
\subsubsection{2.4 单极性码}\label{header-n24}}
定义${a_k}$
\begin{equation}
a_k =\left\{
\begin{aligned}
&1, P(a_k=1) = p\\
&0, P(a_k=0) = 1-p \\
\end{aligned}
\right.
\end{equation}

\begin{equation}
\begin{aligned}
E[a_n] &= 1\times p + 0\times(1-p)=p\\
E[a_n^2] &= 1\times p + 0\times(1-p)=p\\
D[a_n] &= E[a_n^2] - E^2[a_n]=p-p^2\\
\end{aligned}
\end{equation}



对于更一般的情况,$p = 0.5$的条件下,
\begin{equation}
\rho[n] =\left\{
\begin{aligned}
&\frac{1}{2}, n = 0\\
&\frac{1}{4}, n \neq 0 \\
\end{aligned}
\right.
\end{equation}

单极性$NRZ$码的自相关函数为\\
\begin{equation}
\begin{aligned}
R_x(\tau) &= \frac{1}{4 T_b} [\sum\limits_{k=-\infty}^{\infty} \delta(\tau-kT_b) + \delta(\tau)]\\
\end{aligned}
\end{equation}

\begin{equation}
\begin{aligned}
S_x(f) &= \frac{1}{4 T_b} [\frac{1}{T_b} \sum\limits_{k=-\infty}^{\infty} \delta(f-\frac{k}{T_b}) + 1]\\
\end{aligned}
\end{equation}

\begin{equation}
\begin{aligned}
S_{ynrz}(f) &= {\left| S_p(f) \right|}^2 S_x(f) \\
&= \frac{T_b}{4} sinc^2(\pi f T_b) [\frac{1}{T_b}\sum\limits_{k=-\infty}^{\infty} \delta(f-\frac{k}{T_b}) + 1]\\
&= \sum\limits_{k=-\infty}^{\infty} \frac{1}{4} sinc^2(\pi f T_b) \delta(f-\frac{k}{T_b}) + \frac{T_b}{4} sinc^2(\pi f T_b)\\
&= \frac{1}{4} [\delta(f) + T_b sinc^2(\pi f T_b)]\\
\end{aligned}
\end{equation}

单极性$RZ$码的自相关函数为\\
\begin{equation}
\begin{aligned}
S_{yrz}(f) &= {\left| S_p(f) \right|}^2 S_x(f) \\
&= \frac{T_b}{16} sinc^2(\pi f T_b/2) [\frac{1}{T_b}\sum\limits_{k=-\infty}^{\infty} \delta(f-\frac{k}{T_b}) + 1]\\
&= \sum\limits_{k=-\infty}^{\infty} \frac{1}{16} sinc^2(\pi f T_b/2) \delta(f-\frac{k}{T_b}) + \frac{T_b}{16} sinc^2(\pi f T_b/2)\\
&= \sum\limits_{k=-\infty}^{\infty} \frac{1}{16} sinc^2(\pi k/2) + \frac{T_b}{16} sinc^2(\pi f T_b/2)\\
\end{aligned}
\end{equation}

\hypertarget{header-n25}{%
\subsubsection{2.5 双极性码}\label{header-n25}}
定义${a_k}$
\begin{equation}
a_k =\left\{
\begin{aligned}
&1, P(a_k=1) = p\\
&-1, P(a_k=-1) = 1-p \\
\end{aligned}
\right.
\end{equation}

\begin{equation}
\begin{aligned}
E[a_n] &= 1\times p - 1\times(1-p)=2p-1\\
E[a_n^2] &= 1\times p + 1\times(1-p)=1\\
D[a_n] &= E[a_n^2] - E^2[a_n]=4(p-p^2)\\
\end{aligned}
\end{equation}

对于更一般的情况,$p = 0.5$的条件下,
\begin{equation}
\rho[n] =\left\{
\begin{aligned}
1, n = 0\\
0, n \neq 0 \\
\end{aligned}
\right.
\end{equation}

双极性$NRZ$码的自相关函数为\\
\begin{equation}
\begin{aligned}
R_{x}(\tau) &= \frac{1}{T_b} \delta(\tau)\\
\end{aligned}
\end{equation}

\begin{equation}
\begin{aligned}
S_{x}(f) &= \frac{1}{T_b}\\
\end{aligned}
\end{equation}

\begin{equation}
\begin{aligned}
S_{ynrz}(f) &= {\left| S_p(f) \right|}^2 S_x(f) \\
&= T_b sinc^2(\pi f T_b) \\
\end{aligned}
\end{equation}

双极性$RZ$码的功率谱密度函数为\\
\begin{equation}
\begin{aligned}
S_{yrz}(f) &= {\left| S_p(f) \right|}^2 S_x(f) \\
&= \frac{T_b}{4} sinc^2(\pi f T_b/2) \\
\end{aligned}
\end{equation}

\hypertarget{header-n26}{%
\subsubsection{2.6 曼彻斯特码}\label{header-n26}}
曼彻斯特码的$x(t)$可用双极性非归零码表示,但$p(t)$与之前有所不同\\
\begin{equation}
p(t) =\left\{
\begin{aligned}
&1, & -T_b/2 < t < 0\\
&-1, & 0 < t < T_b/2\\
&0, & others \\
\end{aligned}
\right.
\end{equation}
\begin{equation}
\begin{aligned}
p(f) &= i T_b \frac{sin^2(\pi f T_b/2)}{\pi f T_b/2}\\
&= i T_b sinc(\pi f T_b/2) sin(\pi f T_b/2)\\
{\left| S_p(f) \right|}^2 & = T_b^2 sinc^2(\pi f T_b/2) sin^2(\pi f T_b/2)
\end{aligned}
\end{equation}

曼彻斯特码的功率谱密度函数为\\
\begin{equation}
\begin{aligned}
S_{yManchester}(f) &= {\left| S_p(f) \right|}^2 S_x(f) \\
&= T_b sinc^2(\pi f T_b/2) sin^2(\pi f T_b/2) \\
\end{aligned}
\end{equation}

\hypertarget{header-n26}{%
\subsubsection{2.6 N-ASK}\label{header-n26}}
对于$N-ASK$等概率信源,$a_k$的概率分布为

\begin{equation}
\begin{aligned}
P(a_k= \frac{n}{N-1}) = \frac{1}{N}, n \in \{0,\frac{1}{N-1},\cdots,\frac{N-2}{N-1},1\}\\
\end{aligned}
\end{equation}

\begin{equation}
\begin{aligned}
E[a_n] &= \frac{1}{2}\\
E[a_n^2] &= \frac{2N-1}{6(N-1)}\\
D[a_n] &= E[a_n^2] - E^2[a_n]=\frac{2N-1}{6(N-1)} - \frac{1}{4}\\
\end{aligned}
\end{equation}
$T_{sym}$为符号速率
\begin{equation}
\begin{aligned}
T_{sym} = log_2(N) T_b\\
\end{aligned}
\end{equation}

$N-ASK$的功率谱密度函数为
\begin{equation}
\begin{aligned}
S_{N-ASK}(f) &= {\left| S_p(f) \right|}^2 S_x(f) \\
&= T_{sym} sinc^2(\pi f T_{sym}) [(E[a_n^2]- E[a_n]^2) + E[a_n]^2 \frac{1}{T_{sym}} \sum\limits_{\substack{k=-\infty}}^{\infty} \delta(f-\frac{k}{T_{sym}})]\\
&= (E[a_n^2]- E[a_n]^2) T_{sym} sinc^2(\pi f T_{sym}) + E[a_n]^2 \delta(f)\\
&= (E[a_n^2]- E[a_n]^2) log_2(N) T_b sinc^2(\pi f log_2(N) T_b) + E[a_n]^2 \delta(f)\\
&= (\frac{2N-1}{6(N-1)} - \frac{1}{4}) log_2(N) T_b sinc^2(\pi f log_2(N) T_b) + \frac{1}{4} \delta(f)\\
\end{aligned}
\end{equation}


\hypertarget{header-n26}{%
\subsubsection{2.6 nPSK}\label{header-n26}}
\hypertarget{header-n26}{%
\subsubsection{2.6 nQAM}\label{header-n26}}
\hypertarget{header-n26}{%
\subsubsection{2.6 OFDM}\label{header-n26}}


\newpage
\hypertarget{header-n322}{%
\subsection{参考文献}\label{header-n322}}

{[}1{]} 曹志刚, 钱亚生. 现代通信原理{[}M{]}. 清华大学出版社有限公司,
1992.

{[}2{]} CRC16-CCITT by Joe Geluso
\url{http://srecord.sourceforge.net/crc16-ccitt.html#refs}

{[}3{]} CRC calculation, by Sven Reifegerste
\url{http://www.zorc.breitbandkatze.de/crc.html}

{[}4{]} "A Painless Guide to CRC Error Detection Algorithms" by Ross N.
Williams. \url{https://www.zlib.net/crc_v3.txt}

{[}5{]} Cyclic redundancy check
\url{https://en.wikipedia.org/wiki/Cyclic_redundancy_check}

{[}6{]} CRC RevEng
\url{http://reveng.sourceforge.net/crc-catalogue/16.htm#crc.cat.crc-16-kermit}

\end{document}